% === Cours de Java
% === Version couleur pour la présentation

\documentclass[14pt,xcolor,table]{beamer}
\usetheme{Warsaw}

\usepackage{mic}

\title{Microprocesseurs (MIC)}
\subtitle{Chap. ?: Codage des instructions}
\date{} 

% =====================   DOCUMENT ===================++
\begin{document}

% ---------------------   FRAME   --------------------
\begin{frame}
	\titlepage
\end{frame}

% ---------------------   FRAME   --------------------
\begin{frame}[fragile]{Objectifs}
% ----------------------------------------------------
	
	\begin{itemize}
	\item 
		Comprendre comment sont codées les instructions du \verb_x86_
		\bigskip
	\item 
		Être capable de traduire 
		\begin{center}
			Assembleur $\Longleftrightarrow$ Code machine
		\end{center}
	\end{itemize}
	
\end{frame}

% ---------------------   FRAME   --------------------
\begin{frame}[fragile]{Plan}
% ----------------------------------------------------

	\begin{itemize}
	\item Format général d'une instruction
	\item Le code opératoire
	\item Instructions sans opérande
	\item Instructions avec 1 opérande (cas simples)
	\item Les bytes ModR/M et SIB
	\item Modes d'adressages complexes
	\item Instructions avec 2 opérandes
	\item Les préfixes
	\end{itemize}

\end{frame}

% ---------------------   FRAME   --------------------
\begin{frame}[fragile]{Forme générale d'une instruction}
% ----------------------------------------------------

	% -- Format
	\hrule
	\begin{footnotesize}
		\begin{tabular}{cccccc}
			préfixe & code op. & {\scriptsize ModR/M} & SIB & déplacement & val. imméd.\\
			\fcolorbox{black}{lightgray}{\color{lightgray}{X}}\fcolorbox{black}{lightgray}{\color{lightgray}{X}}\fcolorbox{black}{lightgray}{\color{lightgray}{X}}\fcolorbox{black}{lightgray}{\color{lightgray}{X}} &
			\fcolorbox{black}{gray}{\color{gray}{X}}\fcolorbox{black}{lightgray}{\color{lightgray}{X}} &
			\fcolorbox{black}{lightgray}{\color{lightgray}{X}} &
			\fcolorbox{black}{lightgray}{\color{lightgray}{X}} &
			\fcolorbox{black}{lightgray}{\color{lightgray}{X}}\fcolorbox{black}{lightgray}{\color{lightgray}{X}}\fcolorbox{black}{lightgray}{\color{lightgray}{X}}\fcolorbox{black}{lightgray}{\color{lightgray}{X}} &
			\fcolorbox{black}{lightgray}{\color{lightgray}{X}}\fcolorbox{black}{lightgray}{\color{lightgray}{X}}\fcolorbox{black}{lightgray}{\color{lightgray}{X}}\fcolorbox{black}{lightgray}{\color{lightgray}{X}} \\
			1 à 4 & & & & 1, 2 ou 4 & 1, 2 ou 4\\
		\end{tabular}
	\end{footnotesize}
	\hrule
	
	% -- Légende
	\begin{center}
		\begin{scriptsize}
			\fcolorbox{black}{gray}{\color{gray}{X}} = obligatoire \qquad
			\fcolorbox{black}{lightgray}{\color{lightgray}{X}} = facultatif
		\end{scriptsize}
	\end{center}
	
	% -- Remarques
	\begin{itemize}
	\item Longueur variable en fonction de l''instruction
	\item Assez complexe $\Rightarrow$ procédons par étapes
	\end{itemize}
	
\end{frame}

% ---------------------   FRAME   --------------------
\begin{frame}[fragile]{Le code opératoire}
% ----------------------------------------------------

	Identifie l'opération à exécuter
	\begin{itemize}
	\item Souvent sur 1 byte, parfois sur 2
	\item Parfois, utilise une partie du byte ModR/M
		\\(on parle d'extension du code)
	\end{itemize}

	\medskip
	Un mnémonique, plusieurs code associés
	\begin{itemize}
	\item En fonction des opérandes
	\item Ex: \asm{INC} peut se coder 40 à 47, FE ou FF
	\end{itemize}

	\medskip
	Un code opératoire, une seule instruction
	\\(extension comprise)
	\begin{itemize}
	\item Ex: CD est un \asm{INT}, 01 est un \asm{ADD}
	\end{itemize}
	
\end{frame}

% ---------------------   FRAME   --------------------
\begin{frame}[fragile]{Référence}
% ----------------------------------------------------

	Où trouver ces codes opératoires et les détails ?
	\begin{itemize}
	\item Dans une référence (disponible sur poÉSI)
	\item Exemple

	\begin{center}
	\begin{footnotesize}
	\begin{tabular}{|c|c|}
		\multicolumn{2}{c}{\asm{DEC}}\\\hline
		{\cellcolor{gray!25}Opcode} & {\cellcolor{gray!25}Opérandes} \\\hline
		FE /1 & r/m8 \\\hline
		FF /1 & r/m16 \\\hline
		48+rw & r16 \\\hline
		48+rw & r32 \\\hline
	\end{tabular}
	\end{footnotesize}
	\end{center}

	\item
		Pas évident à lire; 
		nous introduirons les notations petit à petit
	\end{itemize}
	
\end{frame}

% ---------------------   FRAME   --------------------
\begin{frame}[fragile]{Instruction sans paramètre}
% ----------------------------------------------------

	Codée sur 1 byte : le code opératoire de l'instruction
	\begin{center}
		\emph{\fbox{code op.}}
	\end{center}

	\bigskip 
	\bulle{Exemple : \asm{NOP}}

	\begin{center}
	\begin{footnotesize}
	\begin{tabular}{|c|c|}
		\multicolumn{2}{c}{\asm{NOP}}\\\hline
		{\cellcolor{gray!25}Opcode} & {\cellcolor{gray!25}Opérandes} \\\hline
		90 & \\\hline
	\end{tabular}
	\end{footnotesize}
	\end{center}

	\begin{center}
		\asm{NOP} $\Longrightarrow \underbrace{\fbox{90}}_{\text{code op.}}$ 
	\end{center}

\end{frame}

% ---------------------   FRAME   --------------------
\begin{frame}[fragile]{Instruction avec une opérande immédiate}
% ----------------------------------------------------

	Codage : {\small\emph{\fbox{code op.} \fbox{immédiat sur 1, 2 ou 4 bytes}}}
	
	\bulle{\asm{INT 0x80}}
	
	\begin{center}
	\begin{footnotesize}
	\begin{tabular}{|c|c|c|}
		\multicolumn{3}{c}{\asm{INT}}\\\hline
		{\cellcolor{gray!25}Opcode} & {\cellcolor{gray!25}Opérandes} & {\cellcolor{gray!25}Explications} \\\hline
		CD ib & imm8 & imm8 est le numéro de l'interruption \\\hline
	\end{tabular}
	\end{footnotesize}
	\end{center}
	
	\begin{itemize}
	\item \emph{ib} : l'opcode est étendu par une valeur immédiate sur un byte
	\item \emph{imm8} : l'opérande est une valeur immédiate interprétée sur 1 byte
	\end{itemize}	
	
	\begin{center}
		\asm{INT 0x80} 
			$\Longrightarrow$ 
			$\underbrace{\fbox{CD}}_{\text{code op.}}$
			$\underbrace{\fbox{80}}_{\text{imm8}}$
	\end{center}
		
\end{frame}

% ---------------------   FRAME   --------------------
\begin{frame}[fragile]{Coder les registres}
% ----------------------------------------------------

	Dans les exemples suivants, nous devrons coder des registres.
	Comment ?
	\begin{itemize}
	\item Via 3 bits
	
		\begin{center}
		\begin{tabular}{|l|l||l|l|}
		\hline 
		AL, AX, EAX & 000 & AH, SP, ESP & 100\tabularnewline
		\hline 
		CL, CX, ECX & 001 & CH, BP, EBP & 101\tabularnewline
		\hline 
		DL, DX, EDX & 010 & DH, SI, ESI & 110\tabularnewline
		\hline 
		BL, BX, EBX & 011 & BH, DI, EDI & 111\tabularnewline
		\hline 
		\end{tabular}
		\end{center}
		
	\item 
		À quel registre correspond un code ?
		\\Dépend du contexte.
	\end{itemize}

\end{frame}

% ---------------------   FRAME   --------------------
\begin{frame}[fragile]{Instruction avec une opérande registre}
% ----------------------------------------------------

	Codage : \emph{\fbox{code op. + \no registre}}
	
	\medskip
	\bulle{\asm{INC EBX}}
	
	\begin{center}
	\begin{small}
	\begin{tabular}{|c|c|c|}
		\multicolumn{3}{c}{\asm{INC} (extrait)}\\\hline
		{\cellcolor{gray!25}Opcode} & {\cellcolor{gray!25}Opérandes} & {\cellcolor{gray!25}Explications} \\\hline
		40 + rd & r32 & incrémente le registre désigné \\\hline
	\end{tabular}
	\end{small}
	\end{center}
	
	\begin{itemize}
	\item \emph{+rd} : le numéro du registre 32 bits est ajouté à l'opcode
	\item \emph{r32} : l'opérande est un registre 32 bits
	\end{itemize}	
	
	\begin{center}
		\asm{INC EBX}
			$\Longrightarrow$ 
			$\underbrace{\fbox{43}}_{\text{40 (code op.) + 3 (EBX)}}$
	\end{center}
	
\end{frame}

% ---------------------   FRAME   --------------------
\begin{frame}[fragile]{Le byte ModR/M}
% ----------------------------------------------------

	Utilisé pour les modes d'adressages complexes
	
	\begin{center}
		{\small ModR/M} \\
		\fbox{
		$\underbrace{\fbox{\ }\fbox{\ }}_{Mod}$
		$\underbrace{\fbox{\ }\fbox{\ }\fbox{\ }}_{divers}$
		$\underbrace{\fbox{\ }\fbox{\ }\fbox{\ }}_{R/M} $
		}
	\end{center}
	
	\begin{itemize}
	\item \emph{Mod} : mode d'adressage pour la partie R/M
	\item \emph{R/M} : opérande \\(peut nécessiter d'autres bytes)
	\item \emph{divers} : dépend du nombre d'opérandes
		\begin{itemize}
		\item 1 opérande : extension du code opératoire
		\item 2 opérandes : deuxième opérande (un registre)
		\end{itemize}	
	\end{itemize}	

\end{frame}

% ---------------------   FRAME   --------------------
\begin{frame}[fragile]{Mode d'adressage plus complexe}
% ----------------------------------------------------

	Analysons la référence de \asm{INC}

	\begin{center}
	\begin{small}
	\begin{tabular}{|c|c|}
		\multicolumn{2}{c}{\asm{INC} (extrait)}\\\hline
		{\cellcolor{gray!25}Opcode} & {\cellcolor{gray!25}Opérandes} \\\hline
		FF /0 & r/m32\\\hline
	\end{tabular}
	\end{small}
	\end{center}
	
	\begin{itemize}
	\item 
		\emph{r/m32} : l'opérande est un registre 32 bits ou une mémoire 32 bits
		\begin{itemize}
		\item utilisation du byte ModR/M
		\item permet les différents adressages complexes
		\item nous allons les voir un à un
		\end{itemize}
	\item 
		\emph{/0} : la partie \textit{divers} de ModR/M vaut 0
		\begin{itemize}
		\item on parle d'extension de code
		\end{itemize}
	\end{itemize}	
	
\end{frame}

% ---------------------   FRAME   --------------------
\begin{frame}[fragile]{Mode d'adressage par registre}
% ----------------------------------------------------

	Codage : \emph{
		\fbox{code op.} 
		\fbox{
			$\fbox{1}\fbox{1}$
			$\underbrace{\fbox{\color{white}{X}}\fbox{\color{white}{X}}\fbox{\color{white}{X}}}_{ext. code}$
			$\underbrace{\fbox{\color{white}{X}}\fbox{\color{white}{X}}\fbox{\color{white}{X}}}_{registre} $
		}
	}
	
	Inutile pour \asm{INC} qui propose une autre forme mais ce n'est pas toujours le cas

	\bulle{\asm{NOT EBX}}
	
	\begin{center}
	\begin{small}
	\begin{tabular}{|c|c|}
		\multicolumn{2}{c}{\asm{NOT} (extrait)}\\\hline
		{\cellcolor{gray!25}Opcode} & {\cellcolor{gray!25}Opérandes} \\\hline
		F7 /2 & r/m32\\\hline
	\end{tabular}
	\end{small}
	\end{center}

	\bigskip
	\quad$\Rightarrow$ 
	$\underbrace{\fbox{F7}}_{code op.}$
	\fbox{
		$\fbox{1}\fbox{1}$
		$\underbrace{\fbox{0}\fbox{1}\fbox{0}}_{/2}$
		$\underbrace{\fbox{0}\fbox{1}\fbox{1}}_{EBX} $
		}
	\quad$\Rightarrow$ 
	\fbox{F7} \fbox{D3}
	\bigskip
	
\end{frame}

% ---------------------   FRAME   --------------------
\begin{frame}[fragile]{Mode d'adressage indirect}
% ----------------------------------------------------

	Codage : \emph{
		\fbox{code op.} 
		\fbox{
			$\fbox{0}\fbox{0}$
			$\underbrace{\fbox{\color{white}{X}}\fbox{\color{white}{X}}\fbox{\color{white}{X}}}_{ext. code}$
			$\underbrace{\fbox{\color{white}{X}}\fbox{\color{white}{X}}\fbox{\color{white}{X}}}_{registre} $
		}
	}
	
	\bulle{\asm{INC dword [EBX]}}
	
	\bigskip
	%\asm{INC [EBX]} 
	\quad$\Rightarrow$ 
	$\underbrace{\fbox{FF}}_{code op.}$
	\fbox{
		$\fbox{0}\fbox{0}$
		$\underbrace{\fbox{0}\fbox{0}\fbox{0}}_{/0}$
		$\underbrace{\fbox{0}\fbox{1}\fbox{1}}_{EBX} $
		}
	\\\medskip\quad$\Rightarrow$ 
	\fbox{FF} \fbox{03}
	\bigskip
		
	Remarque : le registre ne peut être ni \asm{EBP} ni \asm{ESP}
	
\end{frame}

% ---------------------   FRAME   --------------------
\begin{frame}[fragile]{Adressage indirect avec déplacement court}
% ----------------------------------------------------

	\begin{small}
		\emph{
			\fbox{code op.} 
			\fbox{
				$\fbox{0}\fbox{1}$
				$\underbrace{\fbox{\color{white}{X}}\fbox{\color{white}{X}}\fbox{\color{white}{X}}}_{ext. code}$
				$\underbrace{\fbox{\color{white}{X}}\fbox{\color{white}{X}}\fbox{\color{white}{X}}}_{registre} $
			}
			$\underbrace{\fbox{déplacement}}_{\text{1 byte}}$
		}
	\end{small}
	
	\medskip
	\bulle{\asm{INC dword [EBX + 10]}}
	
	\bigskip
	\smallskip\quad$\Rightarrow$ 
	$\underbrace{\fbox{FF}}_{code op.}$
	\fbox{
		$\fbox{0}\fbox{1}$
		$\underbrace{\fbox{0}\fbox{0}\fbox{0}}_{/0}$
		$\underbrace{\fbox{0}\fbox{1}\fbox{1}}_{EBX}$
	}
	$\underbrace{\fbox{0A}}_{\text{dépl.}}$
	\\\smallskip\quad$\Rightarrow$ 
	\fbox{FF} \fbox{43} \fbox{0A}
	\bigskip
		
	Remarque : le registre ne peut pas être \asm{ESP}

\end{frame}

% ---------------------   FRAME   --------------------
\begin{frame}[fragile]{Adressage indirect avec déplacement long}
% ----------------------------------------------------

	\begin{small}
		\emph{
			\fbox{code op.} 
			\fbox{
				$\fbox{1}\fbox{0}$
				$\underbrace{\fbox{\color{white}{X}}\fbox{\color{white}{X}}\fbox{\color{white}{X}}}_{ext. code}$
				$\underbrace{\fbox{\color{white}{X}}\fbox{\color{white}{X}}\fbox{\color{white}{X}}}_{registre} $
			}
			$\underbrace{\fbox{déplacement}}_{\text{4 bytes}}$
		}
	\end{small}

	\medskip
	\bulle{\asm{INC dword [EAX+0x1234]}}
	
	\bigskip
	\begin{small}
		\smallskip\quad$\Rightarrow$ 
		$\underbrace{\fbox{FF}}_{code op.}$
		\fbox{
			$\fbox{1}\fbox{0}$
			$\underbrace{\fbox{0}\fbox{0}\fbox{0}}_{/0}$
			$\underbrace{\fbox{0}\fbox{0}\fbox{0}}_{EAX}$
		}
		$\underbrace{\text{\fbox{34} \fbox{12} \fbox{00} \fbox{00}}}_{\text{dépl. en petit-indien}}$
		\\\smallskip\quad$\Rightarrow$ 
		\fbox{FF} \fbox{80} \fbox{34} \fbox{12} \fbox{00} \fbox{00}
	\end{small}
	\bigskip
		
	Remarque : le registre ne peut pas être \asm{ESP}

\end{frame}

% ---------------------   FRAME   --------------------
\begin{frame}[fragile]{Adressage direct}
% ----------------------------------------------------

	\begin{small}
		\emph{
			\fbox{code op.}
			\fbox{
				$\fbox{0}\fbox{0}$
				$\underbrace{\fbox{\color{white}{X}}\fbox{\color{white}{X}}\fbox{\color{white}{X}}}_{ext. code}$
				$\underbrace{\fbox{1}\fbox{0}\fbox{1}}_{\text{exc. EBP}}$
			}
			$\underbrace{\fbox{adresse}}_{\text{4 bytes}}$
		}
	\end{small}

	\medskip
	\bulle{\asm{INC dword [0x1234]}}
	
	\bigskip
	\begin{small}
		\smallskip\quad$\Rightarrow$ 
		$\underbrace{\fbox{FF}}_{code op.}$
		\fbox{
			$\fbox{0}\fbox{0}$
			$\underbrace{\fbox{0}\fbox{0}\fbox{0}}_{/0}$
			$\underbrace{\fbox{1}\fbox{0}\fbox{1}}_{\text{imposé}}$
		}
		$\underbrace{\text{\fbox{34} \fbox{12} \fbox{00} \fbox{00}}}_{\text{dépl. en petit-indien}}$
		\\\medskip\quad$\Rightarrow$ 
		\fbox{FF} \fbox{05} \fbox{34} \fbox{12} \fbox{00} \fbox{00}
	\end{small}
	\bigskip
	
	% vrai si on compile avec l'option bin mais pas avec le elf32
	%\begin{tiny}
	%Note : valable en mode [BITS 32], en mode [BITS 16] donne FF 06\dots
	%\end{tiny}
	
\end{frame}

% ---------------------   FRAME   --------------------
\begin{frame}[fragile]{Le byte SIB}
% ----------------------------------------------------

	\begin{itemize}
	\item Utilisé en complément à ModR/M
	\item Pour les modes d'adressages indexés {\small ($B + S\times I$)}
	\end{itemize}
	
	\begin{center}
		{\small SIB} \\
		\fbox{
		$\underbrace{\fbox{\ }\fbox{\ }}_{S}$
		$\underbrace{\fbox{\ }\fbox{\ }\fbox{\ }}_{I}$
		$\underbrace{\fbox{\ }\fbox{\ }\fbox{\ }}_{B} $
		}
	\end{center}
	
	\begin{description}
	\item[\emph{S}] facteur multiplicatif (scale)
		\\\quad{\small $2^i \Rightarrow 00=1\times, 01=2\times, 10=4\times, 11=8\times$}
	\item[\emph{I}] registre d'index
	\item[\emph{B}] registre de base
	\end{description}	

\end{frame}

% ---------------------   FRAME   --------------------
\begin{frame}[fragile]{Indirect indexé}
% ----------------------------------------------------

	\begin{small}
		\emph{
			\fbox{code op.}
			\fbox{
				$\fbox{0}\fbox{0}$
				$\underbrace{\fbox{\color{white}{X}}\fbox{\color{white}{X}}\fbox{\color{white}{X}}}_{ext. code}$
				$\underbrace{\fbox{1}\fbox{0}\fbox{0}}_{\text{exc. ESP}}$
			}
			\fbox{SIB}
		}
	\end{small}

	\medskip
	\bulle{\asm{INC dword [EAX+4*EBX]}}
	
	\bigskip
	\begin{scriptsize}
		\smallskip\quad$\Rightarrow$ 
		$\underbrace{\fbox{FF}}_{code op.}$
		{
			\fbox{
				\fbox{0}\fbox{0}
				$\underbrace{\fbox{0}\fbox{0}\fbox{0}}_{/0}$
				$\underbrace{\fbox{1}\fbox{0}\fbox{0}}_{\text{imposé}}$
			}
		}
		\fbox{
			$\underbrace{\fbox{1}\fbox{0}}_{*4}$
			$\underbrace{\fbox{0}\fbox{1}\fbox{1}}_{EBX}$
			$\underbrace{\fbox{0}\fbox{0}\fbox{0}}_{EAX}$
		}
		\\\medskip\quad$\Rightarrow$ 
		\fbox{FF} \fbox{04} \fbox{98}
	\end{scriptsize}
	\bigskip
		
\end{frame}

% ---------------------   FRAME   --------------------
\begin{frame}[fragile]{Indirect indexé avec déplacement court}
% ----------------------------------------------------

	\begin{small}
		\emph{
			\fbox{code op.}
			\fbox{
				$\fbox{0}\fbox{1}$
				$\underbrace{\fbox{\color{white}{X}}\fbox{\color{white}{X}}\fbox{\color{white}{X}}}_{ext. code}$
				$\underbrace{\fbox{1}\fbox{0}\fbox{0}}_{\text{exc. ESP}}$
			}
			\fbox{SIB}
			$\underbrace{\fbox{adresse}}_{\text{1 byte}}$
		}
	\end{small}

	\medskip
	\bulle{\asm{INC dword [EAX+4*EBX+10]}}
	
	\bigskip
	\begin{scriptsize}
		\smallskip\quad$\Rightarrow$ 
		$\underbrace{\fbox{FF}}_{code op.}$
		{
			\fbox{
				\fbox{0}\fbox{1}
				$\underbrace{\fbox{0}\fbox{0}\fbox{0}}_{/0}$
				$\underbrace{\fbox{1}\fbox{0}\fbox{0}}_{\text{imposé}}$
			}
		}
		\fbox{
			$\underbrace{\fbox{1}\fbox{0}}_{\times 4}$
			$\underbrace{\fbox{0}\fbox{1}\fbox{1}}_{EBX}$
			$\underbrace{\fbox{0}\fbox{0}\fbox{0}}_{EAX}$
		}
		$\underbrace{\text{\fbox{0A}}}_{\text{dépl}}$
		\\\medskip\quad$\Rightarrow$ 
		\fbox{FF} \fbox{44} \fbox{98} \fbox{0A}
	\end{scriptsize}
	\bigskip
		
\end{frame}

% ---------------------   FRAME   --------------------
\begin{frame}[fragile]{ModR/M - récapitulatif}
% ----------------------------------------------------

	\begin{small}
	\begin{tabular}{l|l|l}
		Adressage & Exemple & ModR/M \\\hline\hline
		registre & EAX & 
					$\fbox{1}\fbox{1}$
					$\fbox{\color{white}{0}}\fbox{\color{white}{0}}\fbox{\color{white}{0}}$
					$\fbox{\color{white}{0}}\fbox{\color{white}{0}}\fbox{\color{white}{0}}$
			 \\
		indirect & [EAX] &
					$\fbox{0}\fbox{0}$
					$\fbox{\color{white}{0}}\fbox{\color{white}{0}}\fbox{\color{white}{0}}$
					$\fbox{\color{white}{0}}\fbox{\color{white}{0}}\fbox{\color{white}{0}}$
			 \\
		indirect + court & [EAX+10] &
					$\fbox{0}\fbox{1}$
					$\fbox{\color{white}{0}}\fbox{\color{white}{0}}\fbox{\color{white}{0}}$
					$\fbox{\color{white}{0}}\fbox{\color{white}{0}}\fbox{\color{white}{0}}$
			\\
		indirect + long & [EAX+800] &
					$\fbox{1}\fbox{0}$
					$\fbox{\color{white}{0}}\fbox{\color{white}{0}}\fbox{\color{white}{0}}$
					$\fbox{\color{white}{0}}\fbox{\color{white}{0}}\fbox{\color{white}{0}}$
			\\
		direct & [adresse] &
					$\fbox{0}\fbox{0}$
					$\fbox{\color{white}{0}}\fbox{\color{white}{0}}\fbox{\color{white}{0}}$
					$\fbox{1}\fbox{0}\fbox{1}$
			\\
		indirect indexé & [EAX+4*EBX] &
					$\fbox{0}\fbox{0}$
					$\fbox{\color{white}{0}}\fbox{\color{white}{0}}\fbox{\color{white}{0}}$
					$\fbox{1}\fbox{0}\fbox{0}$
			\\
		indexé + court & [EAX+4*EBX+10] &
					$\fbox{0}\fbox{1}$
					$\fbox{\color{white}{0}}\fbox{\color{white}{0}}\fbox{\color{white}{0}}$
					$\fbox{1}\fbox{0}\fbox{0}$
			\\
		indexé + long & [EAX+4*EBX+800] &
					$\fbox{1}\fbox{0}$
					$\fbox{\color{white}{0}}\fbox{\color{white}{0}}\fbox{\color{white}{0}}$
					$\fbox{1}\fbox{0}\fbox{0}$
			\\
	\end{tabular}
	\end{small}
	
\end{frame}

% ---------------------   FRAME   --------------------
\begin{frame}[fragile]{Instruction avec deux opérandes}
% ----------------------------------------------------

	Remarques générales
	\begin{itemize}
	\item Une des deux opérandes sera toujours un registre (mode registre)
	\item Pour l'autre on aura toutes les possibilités
	\end{itemize}
	
	\bigskip
	Exemples
	\begin{itemize}
	\item \asm{ADD EAX, brol} (registre, immédiat) $\Rightarrow$ OK
	\item \asm{ADD [EAX], EBX} (indirect, registre) $\Rightarrow$ OK
	\item \asm{ADD EAX, EBX} (registre, registre) $\Rightarrow$ OK
	\item \asm{ADD [EAX], [EBX]} (indirect, indirect)  $\Rightarrow$ interdit
	\end{itemize}
	
\end{frame}

% ---------------------   FRAME   --------------------
\begin{frame}[fragile]{Instruction avec deux opérandes}
% ----------------------------------------------------

	Le code opératoire va dépendre de l'ordre des opérandes
	\begin{center}
	\begin{small}
	\begin{tabular}{|c|c|c|}
	\multicolumn{2}{c}{\asm{ADD} (extrait)}\\\hline
	{\cellcolor{gray!25}Opcode} & {\cellcolor{gray!25}Opérandes} \\\hline
	01 /r & r32/mem32, r32 \\\hline
	03 /r & r32, r32/mem32 \\\hline
	\end{tabular}
	\end{small}
	\end{center}
	
	\bigskip
	Exemples
	\begin{itemize}
	\item \asm{ADD EAX, brol} utilise le code \verb_03_
	\item \asm{ADD [EAX], EBX} utilise le code \verb_01_
	\item \asm{ADD EAX, EBX} utilise le code \verb_01_ ou \verb_03_
	\end{itemize}
	
\end{frame}

% ---------------------   FRAME   --------------------
\begin{frame}[fragile]{Exemple}
% ----------------------------------------------------

	\bulle{\asm{ADD ECX, [EBX]}}

	\begin{center}
	\begin{small}
	\begin{tabular}{|c|c|c|}
	\multicolumn{2}{c}{\asm{ADD} (extrait)}\\\hline
		{\cellcolor{gray!25}Opcode} & {\cellcolor{gray!25}Opérandes} \\\hline
	01 /r & r32/mem32, r32 \\\hline
	03 /r & r32, r32/mem32 \\\hline
	\end{tabular}
	\end{small}
	\end{center}

	\bigskip
	\quad$\Rightarrow$ 
	$\underbrace{\fbox{03}}_{code op.}$
	\fbox{
		$\underbrace{\fbox{0}\fbox{0}}_{indirect}$
		$\underbrace{\fbox{0}\fbox{0}\fbox{1}}_{ECX}$
		$\underbrace{\fbox{0}\fbox{1}\fbox{1}}_{EBX} $
		}
	\\\medskip\quad$\Rightarrow$ 
	\fbox{03} \fbox{0B}
	\bigskip
	
\end{frame}

% ---------------------   FRAME   --------------------
\begin{frame}[fragile]{Exemple}
% ----------------------------------------------------

	\bulle{\asm{ADD [EAX+10], EBX}}

	\begin{center}
	\begin{small}
	\begin{tabular}{|c|c|c|}
	\multicolumn{2}{c}{\asm{ADD} (extrait)}\\\hline
		{\cellcolor{gray!25}Opcode} & {\cellcolor{gray!25}Opérandes} \\\hline
	01 /r & r32/mem32, r32 \\\hline
	03 /r & r32, r32/mem32 \\\hline
	\end{tabular}
	\end{small}
	\end{center}

	\bigskip
	\quad$\Rightarrow$ 
	$\underbrace{\fbox{01}}_{code op.}$
	\fbox{
		$\underbrace{\fbox{0}\fbox{1}}_{\text{dépl. court}}$
		$\underbrace{\fbox{0}\fbox{1}\fbox{1}}_{EBX}$
		$\underbrace{\fbox{0}\fbox{0}\fbox{0}}_{EAX} $
		}
	\fbox{0A}
	\\\medskip\quad$\Rightarrow$ 
	\fbox{01} \fbox{58} \fbox{0A}
	\bigskip
	
\end{frame}

% ---------------------   FRAME   --------------------
\begin{frame}[fragile]{Exemple}
% ----------------------------------------------------

	\bulle{\asm{ADD EAX, EBX}}

	\begin{center}
	\begin{small}
	\begin{tabular}{|c|c|c|}
	\multicolumn{2}{c}{\asm{ADD} (extrait)}\\\hline
		{\cellcolor{gray!25}Opcode} & {\cellcolor{gray!25}Opérandes} \\\hline
	01 /r & r32/mem32, r32 \\\hline
	03 /r & r32, r32/mem32 \\\hline
	\end{tabular}
	\end{small}
	\end{center}

	\bigskip	
	\begin{small}
		Se code
		$\underbrace{\fbox{01}}_{code op.}$
		\fbox{
			$\underbrace{\fbox{1}\fbox{1}}_{\text{registre}}$
			$\underbrace{\fbox{0}\fbox{1}\fbox{1}}_{EBX}$
			$\underbrace{\fbox{0}\fbox{0}\fbox{0}}_{EAX} $
			}
		$\Rightarrow$ 
		\fbox{01} \fbox{D8}
	
		\bigskip
		ou bien
		$\underbrace{\fbox{03}}_{code op.}$
		\fbox{
			$\underbrace{\fbox{1}\fbox{1}}_{\text{registre}}$
			$\underbrace{\fbox{0}\fbox{0}\fbox{0}}_{EAX}$
			$\underbrace{\fbox{0}\fbox{1}\fbox{1}}_{EBX} $
			}
		$\Rightarrow$ 
		\fbox{01} \fbox{C3}
	\end{small}
	
\end{frame}

% ---------------------   FRAME   --------------------
\begin{frame}[fragile]{Les préfixes}
% ----------------------------------------------------

	Les préfixes modifient le comportement de l'instruction :
	\begin{itemize}
	\item \emph{Répétition} : répétition automatique de l'instruction
	\item \emph{Taille adresse} : modifie la taille par défaut des adresses (16/32)
	\item \emph{Taille registre} : modifie la taille par défaut des registres (16/32)
	\item \emph{Segment} : modifie le segment par défaut
	\end{itemize}
	
	À placer dans cet ordre
	
\end{frame}

% ---------------------   FRAME   --------------------
\begin{frame}[fragile]{Modification du segment}
% ----------------------------------------------------

	Rappel : une adresse est sous la forme \emph{base : offset}

	\begin{itemize}
	\item \emph{base} : est l'adresse de début du segment
	\item Une table reprend les descripteurs de segments
	\item Des registres spécialisés 
		\verb_CS_, \verb_DS_, \verb_ES_, \verb_FS_, \verb_GS_ et \verb_SS_
		donnent l'index dans la table des segments
	\item \emph{offset} : est le déplacement dans le segment 
		(spécifié par le programmeur)
	\end{itemize}
	
\end{frame}

% ---------------------   FRAME   --------------------
\begin{frame}[fragile]{Modification du segment}
% ----------------------------------------------------

	Le segment utilisé dans une instruction est implicite
	\begin{itemize}
		\item \asm{INC [AX]} utilisera le segment des données (DS)
		\item \asm{JMP brol} utilisera le segment du code (CS)
	\end{itemize}
	
	\medskip
	Peut être modifié (ex: \asm{ADD [ES:EAX], EBX})

	\medskip
	Codé dans le préfixe
	\begin{center}
		\begin{tabular}{|c|c|c|} \hline
			2E = CS & 3E = DS & 26 = ES \\\hline
			64 = FS & 65 = GS & 36 = SS \\\hline  
		\end{tabular}
	\end{center}
	
	\medskip
	\asm{ADD [ES:EAX], EBX}
	$\Rightarrow$ 
	\fbox{26} \fbox{01} \fbox{18}
	
\end{frame}

% ---------------------   FRAME   --------------------
\begin{frame}[fragile]{Préfixe de répétition}
% ----------------------------------------------------

	Permet de répéter \verb_ECX_ fois une instruction
	\begin{itemize}
	\item En assembleur, on préfixe par \asm{REP}
	\item Codé \verb_F3_
	\end{itemize}
	
	Exemple : Répétition de l'instruction \asm{MOVSB} 
	\begin{itemize}
		\item \asm{MOVSB} copie un byte de \verb_[ESI]_
		vers \verb_[EDI]_ puis incrémente \verb_ESI_ et \verb_EDI_.
		\item Donc, pour copier les 10 premiers bytes de chaine1 dans chaine2
		\begin{Asm}
			MOV ESI, chaine1
			MOV EDI, chaine2
			MOV ECX, 10
			REP MOVSB           ; codé F3 A4
		\end{Asm}
	\end{itemize}
\end{frame}

% ---------------------   FRAME   --------------------
\begin{frame}[fragile]{Taille des des données et des adresses}
% ----------------------------------------------------

	Extrait de la référence de ADD
	\begin{center}
	\begin{small}
	\begin{tabular}{|c|c|c|}
		\multicolumn{2}{c}{\asm{ADD} (extrait)}\\\hline
			{\cellcolor{gray!25}Opcode} & {\cellcolor{gray!25}Opérandes} \\\hline
		00 /r & r8/m8, r8 \\\hline
		01 /r & r16/m16, r16 \\\hline
		01 /r & r32/m32, r32 \\\hline
	\end{tabular}
	\end{small}
	\end{center}

	Mais alors, \asm{ADD EAX, EBX} et \asm{ADD AX, BX} se codent de la même façon ?
\end{frame}

% ---------------------   FRAME   --------------------
\begin{frame}[fragile]{Taille des des données et des adresses}
% ----------------------------------------------------

	Un bit \verb_D_ dans le descripteur de segment indique
	si on utilise des registres/adresses 
	16 bits (\verb_D=0_) ou 32 bits (\verb_D=1_)
	
	\bigskip 
	Spécifié en assembleur via les macros 
	\asm{[BITS 16]} et \asm{[BITS 32]}

	\bigskip
	Pour une instruction donnée, 
	peut être modifié pour via les préfixes
	\begin{itemize}
		\item \verb_0x66_ : pour la taille d'une donnée
		\item \verb_0x67_ : pour la taille d'une adresse
	\end{itemize}
	
\end{frame}

% ---------------------   FRAME   --------------------
\begin{frame}[fragile]{Taille des des données et des adresses}
% ----------------------------------------------------

	Exemples
	\begin{Asm}[basicstyle=\small]
	[BITS 16]
	INC AX             ; pas de préfixe
	INC	EAX            ; 0x66 : donnée sur 32 bits
	INC word [EAX]     ; 0x67 : adresse 32 bits
	INC dword [EAX]    ; 0x66 0x67 : donnée et adresse 32 bits
	ADD AX, BX         ; pas de préfixe
	ADD EAX, EBX       ; 0x66 : données sur 32 bits
	ADD EAX, [EBX]     ; 0x66 0x67 : donnée et adresse 32 bits
	\end{Asm}
	
\end{frame}

% ------------------------------------------------------
\end{document}
